\section{Results}
\subsection{Litter stoichiometry and micronutrient content}

The chemical characteristics of the four litter types was previously reported by \cite{Wanek2011}. Initial Macro- and Micro-nutrient content of litter are presented in figure \ref{fig:litchem_h1}, table \ref{tab:summary} gives a summary of differences between litter types. Mean initial C:N ratios were between 1:41 and 1:58, initial C:P ratios between 1:700 and 1:1300. Initial N:P ratios ranged between 1:15 and 1:30. No significant changes occurred during litter incubation except a slight decrease of the C:N ratio (1:41.8 to 1:37.4) found in the most active litter type (SW) after 15 month.

Fe content were more than twice as high for OS (approx. 450 ppm) than for other litter types (approx. 200 ppm). Litter Mn also was highly variable between litter types, ranging between 170 and 2130 ppm. Litter Mn content was negatively correlated to N ratio [stat]. Changes of micro-nutrient concentrations during litter incubation were significant, but in all cases \textless 15\% of the initial concentration.

Soluble organic carbon content decreased between the first three harvests (14 to 181 days), to strongly increase after 375 days. Contents ranged between 0.1 and 0.7 mg/g d.w. after 14, 97 and 181 days, and increased to amounts between 1.5 and 4 mg/g after 375 days. After 14 and 97 days, the highest C content was found in SW litter followed by AK (see fig. \ref{fig:doc}. DOC content was loosely correlated to litter N content after 14 (R=0.69, p=***) and 97 days (R = 0.65, p =**), they were strictly correlated after 181 days (R = 0.85, p=***) and 375 days (R=0.9, p=***). 

\subsection{Decomposition Processes}

\subsubsection{Litter mass loss and respiration}

Litter mass loss was not significant after 2 weeks and 3 month, significant for 2 litter types after 6 month. After 15 month, litter mass loss was significant for all litter types, and strongly correlated to litter N content (R=0.794, p=***). Detailed results were reported by \citep{Mooshammer2011}. After 15 month, between 5 and 12\% of the initial dry mass was lost. This is less than reported in litter decomposition studies on other species, but in a similar range as recently reported for beech litter from an in-situ litterbag-study \citep{Kalbitz2006} .

Highest respiration rates were measured after 14 days incubation (150-350 \textmu g CO\textsubscript{2}-C d-1 g-1 litter-C), dropped to rates between between 75 and 100 \textmu g CO\textsubscript{2}-C d-1 g-1 litter-C after 97 days. After 181 and 375 days, respiration rates for AK and OS further decreased, while SW and KL show a second maximal respiration after 181 days. [make graph!] Respiration was correlated to litter N content after 2 weeks, 6 and 15 month(R\textgreater 0.70 in all cases, all p=***), but not after 3 month. All harvests combined were weakly correlated to litter N content (R=0.416, p=***).

Accumulated respiration was correlated to litter mass loss for all harvest with significant mass loss when means per litter type and harvest were compared (n=6) [statistics]. Slope was [\textless 1] indicating a general underestimation of CO2-C [check]. Nevertheless, due to the high correlations to mass loss after 6 and 15 month, we assume that the amounts of accumulated respiration calculated allow comparing litter decomposition rates between different harvests and litter types.

\subsection{Microbial biomass abundance and stoichiometry}

Only minimal amounts of microbial carbon were detected after 14 days (1-2 mg micr.-C g-1 d.w.). Microbial carbon is significantly higher for SW than for other sites between 97 and 375 days (4-6 mg micr.-C g-1 d.w.). While KL and OS show no significant changes between 97 and 375 days (3 mg micr.-C g-1 d.w.), AK shows a distinct maximum after 97 days (3.5 mg micr.-C g-1 d.w.). SW and AK carbon content falls after 375 days, while OS and KL stay at a constant organic C content (fig. \ref{fig:bm} A). Litter nitrogen follows a similar trend: SW is highest during all harvests, AK shows a distinct maximum after 97 days close to SW, but has the lowest microbial N content during all other harvests (fig. \ref{fig:bm} B). AK has the highest microbial P content after 97 and 181 days, but drops to the lowest value after 375 days. SW microbial P content continuously increases between 97 and 375 days, while in OS and KL microbial P remains constant between 97 and 375 days (fig. \ref{fig:bm} C). 

After 97 and 181 days, the microbial C:N ratio is highest in SW, after 181 days they are lowest in AK. OS and Kl have intermediate C:N ratios. After 375 days, the relation between litter types turn around: SW has the most narrow C:N ratio, AK the widest (fig. \ref{fig:bm} B). Microbial C:P ratios are ranging between 1:10 and 1:35, widest in SW and most narrow in AK (fig. \ref{fig:bm} D). The N:P ratio remains between 1:1 and 1:2 throuout most of the experiment with slightly higher ratios for AK and SW than for KL and OS. 

\section{Microbial Community}
[Fungi to bacteria ratio highest in SW and lowest in AK, with 

\subsubsection{Potential enzyme activities}

Absolute potential enzyme activities were generally correlated to litter N, respiration and other other decomposition processes [stat[. For all enzymes and at all time points, SW showed the highest and AK the lowest activity. After 14 days, only minimal activities could be detected. Cellulases activity is highest after 3 month and decreases between 97 and 181 days. Peroxidase and Peroxidase activities reach their maximum after 181 days and were highly correlated to each other (fig. \ref{fig:enz}).. After between 6 and 15 month, cellulase activity strongly increased. After 375 days, the activity of oxidative enzymes was below the detection limit [data not shown]

The ratio between the potential activities of cellulases and oxidative enzymes was lowest for AK at all time points. Microbial communities in AK litter invest more energy and nitrogen into degrading lignin and less into degrading carbohydrates than other litter types. (fig. \ref{fig:enz})

% \subsubsection{Glucan depolymerization and Respiration}
% 
% Results are presented in fig. \ref{fig:resp_depoly}. Complete data is available only for H2 and H3.  After 97 days, the ratio between respiration and glucose depolymerization is highest in AK and lowest in SW, while after 181 days, it is correlated [stat!] to litter N content, i.e. highest for SW and lowest for AK and OS.
%  
\subsection{Pyrolysis - GC/MS}

%Each group of pyrolysis products was followed by the sum of the peaks of the group.
To verify that the sum of TIC peak areas represents a general trend for all substances in the group, we calculated the correlation both between the sum of a group and the first principal component of all peaks. For all groups except lignin the first principal component represents at least 84\% of the total variance within the group and correlated to the sum of the peaks with R\textgreater 0.99. Only for lignin peaks, only 55\% of variance are explained by the first principal component and it is correlated to the sum of lignin pyrolysis products with R=0.9. All correlations are highly significant (p=***). 

To quantify the lignin to carbohydrates ratio, an index \emph{LCI} was calculated: 

\begin{equation}
LCI = A_{Lignin} / (A_{Lignin} + A_{Carbohydrates})
\end{equation}

were $A_{Lignin}$ and $A_{Carbohydrates}$ are the sums of relative peak areas for lignin and carbohydrate marker, respectively.

Lipophilic substances, especially saturated fatty acids were prominently present in pyrograms of (ADF) lignin fractions [supplementary date? data not shown?]. In bulk litter pyrograms, we identified 6 n-alkyl alkanes and alkenes (C\textsubscript{25}-\textsubscript{C29} odd-chain), a diterpene identified as phytol (C\textsubscript{20}H\textsubscript{40}O) by the NiSt database, and 3 saturated fatty acids with abundances \textgreater 1\% TIC. Furthermore, we found a number of unspecific pyrolysis products, mainly aliphatic aldehydes and alcohols). A detailed list of the pyrolysis products identified can be found in [appendix table1? supplementary material].

\subsubsection{Differences between litter from different sites}

Generally, we found only minor changes in pyrograms during decomposition. A PCA performed on relative peak areas of 128 peaks shows that samples cluster according to litter types, with no constistent seperation between different harvests. \ref{fig:pca.all}. 118 (94.5\%) [!check nr!] of the peaks integrated show significant differences in relative peak areas between different litter types before incubation. LCI for initial litter are similar for all AK, KL and SW, and slightly lower for OS.

[more detailed results? - phenoles content?]

%Comparing the sums of compound classes between litter types shows great convergence at the levels of pyrolysis product groups: 

Some details are worth mentioning here: Within the carbohydrates group, AK and OS have significantly higher peak areas for most (10 of 15) furan-type carbohydrate pyrolysis products, SW and KL are significantly higher in 2, rest show no clear pattern. Cyclopentenone-type carbohydrate pyrolysis products do not show this pattern. 

Among the lignin derived pyrolysis products, while most other peaks are tightly correlated to each other, the ratio between methylguaiacol and guaiacol shows strong differences between AK and OS (0.7:1) vs. KL and SW (0.45:1). Similar differences can be found in the Methylsyringol:syringol ratio. These differences remain constant during litter incubation. Unlike other studies [lit.], we do not find a shift in the guaiacol/syringol ratio during decomposition.

%In a PCA, factor 1 explains differences between different litter types, but is not correlated to decomposition trends or inter-replicate variance. Factor 2 and 3 represent variance between replicas. Only for KL and SW, a decomposition trend can be demonstrated.

\subsubsection{Decomposition trends}

To balance for initial differences in litter composition, for each peak in each sample, we substrate the mean of the relative peak area of the respective peak in initial litter of the litter type. A PCA calculated with the results allows us to demonstrate shifts between pyrolysis products during litter decomposition (fig. \ref{fig:pca.dif}). The first to principal components represent ~45\% of the total variance. Initial litter samples cluster cluster in the bottom right corner of the graph with positive loadings on PCA 1 and negative loadings on PCA2. Decomposed samples are shifted versus fresh litter along different axis: While decomposed SW samples are in the bottom left quadrant of the samples, shifted along PCA 1 toward more negative values and indifferent along PCA2, decomposed AK samples are shifted along PCA2 towards more positive values and do not shift along PCA1. KL and OS show intermediate decomposition trends. Their decomposed samples are placed in the top left corner, combining both decomposition trends.  
Pyrolysis products that are positioned in the bottom-right quadrant  are depleted in all litter types, while products in the top left quadrant are accumulated in all litter types. Substances in the bottom left quadrant are depleted in AK and accumulated in SW, substances in the top-right quadrant show the opposite trend. 
Most lignin markers have negative loadings on PCA1 and PCA2, indicating accumulation in SW and depletion in AK. 

Figure \ref{fig:timeseries} shows shifts in pyrolysis products relative to incubation time and accumulated respiration. Lignin contents were rising and carbohydrate contents decreasing for all litter types except AK. The two litter types with the highest lignin content show a (non-significant) decrease between 6 an 15 month harvests. Non-lignin phenolic pyrolysis products increase for all litter types, with SW's phenols showing increasing less then other litter.

While KL, OS and SW all accumulate lignin at a similar rate relative to dry mass loss/accumulate respiration, AK show no sign of lignin accumulation during early litter decomposition. A lignin maximum was found after 6 month, with relative depletion of lignin (not significant) between 6 and 15 month harvests in two litter types and no further increase of lignin content in the other two sites. 

While the other three sites had a similar increases in lignin:(lignin+carbohydrate) ratio (relative to the respiration rate), no increase in lignin (absolute or relative to carbohydrates was observed). Fig. \ref{fig:lci} (left)

Fig. \ref{fig:car_lig_6month} shows lignin and carbohydrate differences after 6 month. Lignin accumulation is highest in SW and lowest in AK. The the other two sites are inbetween, but only AK and SW are significantly seperated. Carbohydrates are significantly less depleted in AK than in KL, OS and SW. 

To discriminate between lignin accumulation because of higher or lower litter turnover and different substrate preferences, we compared changes in pyrolysis products to accumulated respiration. LCI index is rising in all litter types except AK, where it fell insignificantly.
Fig. \ref{fig:lci} (right)

The lipophilic compounds found show different trends: Alkanes and alkenes show a drastic increase (+80\%) during the first three month. This increase can not be explained by passive accumulation. Unlike alkene, alkanes are decomposed between month 6-15. The unknown compound at RT 20.00 and fatty acids are depleted during litter decomposition, i.e. decomposed faster than average litter biomass (fig \ref{fig:waxes}), and are decomposed faster in N-poor than in N-rich litter. 


%\subsubsection{Fatty acids and aliphatic pyrolysis products}
%Fatty acids are the most comon origin of aliphatic pyrolysis products. (decarboxylisatio)



